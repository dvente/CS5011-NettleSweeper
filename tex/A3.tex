\documentclass[british]{article}
\usepackage{babel}
\usepackage[margin=1in]{geometry}
\usepackage{natbib}
\usepackage{amsmath}
\usepackage{graphicx}
\usepackage{listings}
\usepackage{enumerate}
\newcommand{\code}[1]{\texttt{#1}}
\newtheorem{defin}{Definition}
\newtheorem{prop}{Proposition}
\newtheorem{col}{Corollary}
\newtheorem{thm}{Theorem}
\setlength{\parskip}{1em}


\title{CS5011 A3 Report}
\author{170008773}
\date{\today}
\begin{document}
\maketitle


\section{Parts completed}
\label{completed}
\begin{itemize}
\item We successfully implemented all the requirements for part 1
\item We again, sucessfully implemented all the requirements for part 2.
\end{itemize}

\section{Parts not completed}
\label{uncompleted}
\begin{itemize}
\item At time of writing we did not attempt to implement the SAT-solver strategy.
\end{itemize}

\section{Literature review}
\label{litrev}


\paragraph{The history}

\paragraph{The rules} Minesweeper consists of a rectangular board of cells. At the start of the game, all the cells are covered, and some cells will contain mines. The Player/agent can perform two actions in this game: Flagging or uncovering a cell. If a cell containing a mine is uncovered the agent has lost the game. If a cell that does not contain a mine is uncovered it will reveal a number. This number is equal to the number of cells that are adjacent to the uncovered cell and contain a mine. If a cell is uncovered that is not adjacent to any mines, all of its neighbours will be uncovered. The agent has won when all of the cells that do not contain a mine are uncovered. 

\paragraph{P vs. NP} Much has been written about the complexity of minesweeper. Complexity is a measure of how ``hard" a problem is. \cite{Kaye00} tells us that complexity-theory is a way of extimating the amount of time needed to solve a problem given the \textit{length} of the input. The first class of problems is a class called P, for \textit{polynomial-time computable} problems. These are the porblems that when given an input of length $n$, that can be solved in $n^k$ steps for some exponent $k$. \citeauthor{Kaye00} writes that these probelms are precicely the ones that are practically solvable. Conversely NP or \textit{Nondeterministic Polynomail-time computable} is a class of problems that is solvable in polynomial time using ``non-deterministic" algortihms (i.e. algorithms where the computer is allowed to make some guesses).  

\paragraph{The complexity of minesweeper} \cite{Kaye00} proved that minesweeper is NP-Complete. This 
 
\section{Design}
\label{design}


\section{Examples and Testing}
\label{sec:testing}
 
\subsection{Testing}
\label{subsec:testing}
\paragraph{Initial testing} During the early stages of developemtn we mainly used two forms of testing. Manual inspection of states and outputs and \code{assert} statements. The \code{assert} served as micro unit tests, making sure that the things that worked still worked. Furthermore we visually inspected most of the output and states of the agent and the strategy to verify that the programms worked correctly. 

\paragraph{Framework} After most of the strategies and game logic had been implemented, we implemented a way to automatically run tests with different implemetnations, and print the results in a readable format. This then allowed us to compare results across both the algorithms and maps which allowed us to correct several bugs in the logic of the game, agent and strategies. 
\subsection{Examples}
\label{example}
\paragraph{A single run} A single run of the programm using the easy equation strattegy looks as follows:
\begin{lstlisting}
java -jar Logic2.jar ../worlds/easy/nworld1
		Starting new game
		Probing: (0,0)
		  0  ?  ?  ?  ?
		  ?  ?  ?  ?  ?
		  ?  ?  ?  ?  ?
		  ?  ?  ?  ?  ?
		  ?  ?  ?  ?  ?

		Probing: (1,0)
		  0  0  ?  ?  ?
		  ?  ?  ?  ?  ?
		  ?  ?  ?  ?  ?
		  ?  ?  ?  ?  ?
		  ?  ?  ?  ?  ?

		Probing: (2,0)
		  0  0  0  ?  ?
		  ?  ?  ?  ?  ?
		  ?  ?  ?  ?  ?
		  ?  ?  ?  ?  ?
		  ?  ?  ?  ?  ?

		Probing: (3,0)
		  0  0  0  2  ?
		  ?  ?  ?  ?  ?
		  ?  ?  ?  ?  ?
		  ?  ?  ?  ?  ?
		  ?  ?  ?  ?  ?

		Probing: (1,1)
		  0  0  0  2  ?
		  ?  0  ?  ?  ?
		  ?  ?  ?  ?  ?
		  ?  ?  ?  ?  ?
		  ?  ?  ?  ?  ?

		Probing: (0,1)
		  0  0  0  2  ?
		  0  0  ?  ?  ?
		  ?  ?  ?  ?  ?
		  ?  ?  ?  ?  ?
		  ?  ?  ?  ?  ?

		Probing: (0,2)
		  0  0  0  2  ?
		  0  0  ?  ?  ?
		  1  ?  ?  ?  ?
		  ?  ?  ?  ?  ?
		  ?  ?  ?  ?  ?

		Probing: (1,2)
		  0  0  0  2  ?
		  0  0  ?  ?  ?
		  1  2  ?  ?  ?
		  ?  ?  ?  ?  ?
		  ?  ?  ?  ?  ?

		Probing: (2,1)
		  0  0  0  2  ?
		  0  0  0  ?  ?
		  1  2  ?  ?  ?
		  ?  ?  ?  ?  ?
		  ?  ?  ?  ?  ?

		Probing: (3,1)
		  0  0  0  2  ?
		  0  0  0  2  ?
		  1  2  ?  ?  ?
		  ?  ?  ?  ?  ?
		  ?  ?  ?  ?  ?

		Probing: (2,2)
		  0  0  0  2  ?
		  0  0  0  2  ?
		  1  2  1  ?  ?
		  ?  ?  ?  ?  ?
		  ?  ?  ?  ?  ?

		Probing: (3,2)
		  0  0  0  2  ?
		  0  0  0  2  ?
		  1  2  1  2  ?
		  ?  ?  ?  ?  ?
		  ?  ?  ?  ?  ?

		SPS
		Checking Cell (1,2)
		Checking Cell (2,2)
		Checking Cell (3,2)
		Checking Cell (3,1)
		Checking Cell (3,0)
		Flagging: (4,0)
		  0  0  0  2  F
		  0  0  0  2  ?
		  1  2  1  2  ?
		  ?  ?  ?  ?  ?
		  ?  ?  ?  ?  ?

		Flagging: (4,1)
		  0  0  0  2  F
		  0  0  0  2  F
		  1  2  1  2  ?
		  ?  ?  ?  ?  ?
		  ?  ?  ?  ?  ?

		SPS
		Checking Cell (1,2)
		Checking Cell (2,2)
		Checking Cell (3,2)
		Checking Cell (3,1)
		Probing: (4,2)
		  0  0  0  2  F
		  0  0  0  2  F
		  1  2  1  2  1
		  ?  ?  ?  ?  ?
		  ?  ?  ?  ?  ?

		SPS
		Checking Cell (1,2)
		Checking Cell (2,2)
		Checking Cell (3,2)
		Checking Cell (4,2)
		Probing: (3,3)
		  0  0  0  2  F
		  0  0  0  2  F
		  1  2  1  2  1
		  ?  ?  ?  2  ?
		  ?  ?  ?  ?  ?

		Probing: (4,3)
		  0  0  0  2  F
		  0  0  0  2  F
		  1  2  1  2  1
		  ?  ?  ?  2  0
		  ?  ?  ?  ?  ?

		Probing: (3,4)
		  0  0  0  2  F
		  0  0  0  2  F
		  1  2  1  2  1
		  ?  ?  ?  2  0
		  ?  ?  ?  2  ?

		Probing: (4,4)
		  0  0  0  2  F
		  0  0  0  2  F
		  1  2  1  2  1
		  ?  ?  ?  2  0
		  ?  ?  ?  2  0

		SPS
		Checking Cell (1,2)
		Checking Cell (3,4)
		Flagging: (2,3)
		  0  0  0  2  F
		  0  0  0  2  F
		  1  2  1  2  1
		  ?  ?  F  2  0
		  ?  ?  ?  2  0

		Flagging: (2,4)
		  0  0  0  2  F
		  0  0  0  2  F
		  1  2  1  2  1
		  ?  ?  F  2  0
		  ?  ?  F  2  0

		SPS
		Checking Cell (1,2)
		Checking Cell (2,2)
		Probing: (1,3)
		  0  0  0  2  F
		  0  0  0  2  F
		  1  2  1  2  1
		  ?  3  F  2  0
		  ?  ?  F  2  0

		SPS
		Checking Cell (1,2)
		Flagging: (0,3)
		  0  0  0  2  F
		  0  0  0  2  F
		  1  2  1  2  1
		  F  3  F  2  0
		  ?  ?  F  2  0

		Probing: (0,4)
		  0  0  0  2  F
		  0  0  0  2  F
		  1  2  1  2  1
		  F  3  F  2  0
		  1  ?  F  2  0

		Probing: (1,4)
		  0  0  0  2  F
		  0  0  0  2  F
		  1  2  1  2  1
		  F  3  F  2  0
		  1  3  F  2  0

			Final number of random guesses: 0
			Final number of probes: 20
			Final number of flags: 5
			Number of runs untill success: 1

\end{lstlisting}
 Whereas a run form the ProduceExperimentReport.jar looks like this:
 \begin{lstlisting}
 flags
                          EASY_EQUATION,  RANDOM_GUESS,  SINGLE_POINT,
../worlds/easy/nworld1              5,            5,            5,
../worlds/easy/nworld2              9,            5,            8,
../worlds/easy/nworld3              8,            5,            7,
../worlds/easy/nworld4              7,            5,            7,
../worlds/easy/nworld5              8,            5,            7,
../worlds/hard/nworld1             20,            0,           20,
../worlds/hard/nworld2             34,            0,           32,
../worlds/hard/nworld3             33,            0,           35,
../worlds/hard/nworld4             34,            0,           34,
../worlds/hard/nworld5             34,            0,           34,
../worlds/medium/nworld1           16,            0,           16,
../worlds/medium/nworld2           10,            0,           10,
../worlds/medium/nworld3           16,            0,           17,
../worlds/medium/nworld4           10,            0,           10,
../worlds/medium/nworld5           16,            0,           16,

probes
                          EASY_EQUATION,  RANDOM_GUESS,  SINGLE_POINT,
../worlds/easy/nworld1             20,           20,           20,
../worlds/easy/nworld2             16,           20,           17,
../worlds/easy/nworld3             17,           20,           18,
../worlds/easy/nworld4             18,           20,           18,
../worlds/easy/nworld5             17,           20,           18,
../worlds/hard/nworld1             80,           67,           80,
../worlds/hard/nworld2             66,           11,           68,
../worlds/hard/nworld3             67,           45,           65,
../worlds/hard/nworld4             66,           36,           66,
../worlds/hard/nworld5             66,           25,           66,
../worlds/medium/nworld1           65,           19,           65,
../worlds/medium/nworld2           71,           57,           71,
../worlds/medium/nworld3           65,           34,           64,
../worlds/medium/nworld4           71,           50,           71,
../worlds/medium/nworld5           65,           65,           65,

randomGuesses
                          EASY_EQUATION,  RANDOM_GUESS,  SINGLE_POINT,
../worlds/easy/nworld1              0,            4,            0,
../worlds/easy/nworld2              0,            6,            6,
../worlds/easy/nworld3              0,            7,            1,
../worlds/easy/nworld4              0,            4,            0,
../worlds/easy/nworld5              0,            4,            2,
../worlds/hard/nworld1              0,            7,            5,
../worlds/hard/nworld2              0,            1,            5,
../worlds/hard/nworld3              0,            1,            1,
../worlds/hard/nworld4              0,            1,            0,
../worlds/hard/nworld5              0,            6,            0,
../worlds/medium/nworld1            0,            2,            0,
../worlds/medium/nworld2            0,            1,            0,
../worlds/medium/nworld3            0,            6,            2,
../worlds/medium/nworld4            0,            2,            0,
../worlds/medium/nworld5            0,            9,            1,

runsUntilSuccess
                          EASY_EQUATION,  RANDOM_GUESS,  SINGLE_POINT,
../worlds/easy/nworld1              1,           72,            1,
../worlds/easy/nworld2              1,          345,            3,
../worlds/easy/nworld3              1,          523,            5,
../worlds/easy/nworld4              1,          161,            1,
../worlds/easy/nworld5              1,          137,            3,
../worlds/hard/nworld1              1,         1000,            4,
../worlds/hard/nworld2              1,         1000,            6,
../worlds/hard/nworld3              1,         1000,            2,
../worlds/hard/nworld4              1,         1000,            1,
../worlds/hard/nworld5              1,         1000,            1,
../worlds/medium/nworld1            1,         1000,            1,
../worlds/medium/nworld2            1,         1000,            1,
../worlds/medium/nworld3            1,         1000,           19,
../worlds/medium/nworld4            1,         1000,            1,
../worlds/medium/nworld5            1,         1000,            3,


 \end{lstlisting}
 
\section{Running}
\label{running}
\begin{enumerate}
\item  Several \code{.jar} files are included with the submition. All of the \code{LogicN.jar} files should be run in the same maner: \code{java -jar LogicN.jar <testDirectory>} The program expects there to be a file in this directory called \code{map.txt}. The format of this file is as follows. The first three lines of the file should contain just one integer. The first two should be the length and width of the world respectively. The third should be the number of nettles present in the world. Then the array of the world should follow in CSV format (i.e. rows of integers seperated by commas and rows should be seperated by newlines). For example: \begin{lstlisting} 5
5
5
0, 0, 0, 2, -1
0, 0, 0, 2,-1
1, 2, 1, 2, 1
-1, 3, -1, 2, 0
1, 3, -1, 2, 0
 \end{lstlisting} Further examples of the file and directory structure that the programmes expect are included.
\item There is another \code{.jar} file included with the submition called \code{ProduceExperimentReport.jar}. This file expects as argument the root directory of the experiments. It will then recursively go through this directory tree looking for files called \code{map.txt}  and running the experiments it finds with all provided implementations and record the data those experiments report. When all the experiments are done it will output the result in a table format (one for every variable)   
\end{enumerate}
 
\section{Evaluation}
\label{evals}


\section{Conclusion}
\label{conclusion}

 
 
word count:
\bibliography{/cs/home/dav/Tex/library}{}
\bibliographystyle{apa}
\end{document}